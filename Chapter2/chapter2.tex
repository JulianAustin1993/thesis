%!TEX root = ../thesis.tex
%*******************************************************************************
%*********************************** Second Chapter *****************************
%*******************************************************************************

\chapter{Data sets\label{cha:data}}  %Title of the Second Chapter

\ifpdf
    \graphicspath{{Chapter2/Figs/Raster/}{Chapter2/Figs/PDF/}{Chapter2/Figs/}}
\else
    \graphicspath{{Chapter2/Figs/Vector/}{Chapter2/Figs/}}
\fi

In the following chapter we describe in detail our data which we will use as a source for assessing the performance of the models described within. We use a publicly available set of climate model simulations known as the CESM Large Ensemble (CESM-LE) data set, \citep{kay_community_2015}. The CESM-LE data set provides a good example of EO data that is discussed in Section~\ref{sec:eo}. This data set will be used throughout this body of work as a intriguing example of the abilities of the discussed methodology. 

 \section[CESM-LE]{\label{sec:cesmle}Community Earth System Model - Large Ensemble,  \citep{kay_community_2015}}
 
 The CESM-LE data set is an extremely popular and significant data set in the climate research community. It was developed to enable the assessment of recent past and near future climate change in the presence of internal climate variability, \citep{kay_community_2015}. It does so by providing 40 simulations of a complex climate model where each simulation is subject to the same radiative forcing scenario but begin a slightly perturbed atmospheric state. As such the forty resultant simulations present the various trajectories the model might take due to internal climate variability of the model. 
 
 The model used to run the forty member ensemble is the Community System Earth Model version 1, \citep{hurrell_community_2013}, with the Community Atmosphere model version 5, \citep{hurrell_community_2013}, as the atmospheric component. The model is a fully coupled climate model which consists of a model for each the Land, Ocean, Atmosphere and Sea Ice components of the climate. These are brought together with a coupler model. Figure~\ref{fig:cesm} provides a simple overview as to how CESM model couples the various components. Such a model is capable of simulating various Land, Ocean, Atmosphere and Sea Ice variables of the climate, such as the wind speed, temperature or pressure. The CESM-LE produces simulations of such variables on the nominal $1\deg$ horizontal separation across the globe which induces our spatial resolution of the data.  The ensemble produces results at varying levels of temporal resolution between the years 1920 and 2100 for non-control simulations. The temporal resolution varies by variable of interest between 6-hourly, Daily, and Monthly frequency of observations.
 
 \begin{figure}[htbp!] 
 	\centering    
 	\includegraphics[width=1.0\textwidth]{Example_image}
 	\caption[CESM component models]{The component models for the full CESM model, \citep{kay_community_2015}.}
 	\label{fig:cesm}
 \end{figure}

 
 For this body of work we use the CESM-LE data by considering the forty members as separate simulations. Each simulation giving us a realisation of the various climate variables generated by the process described in \citep{kay_community_2015}. Further we only consider modelling the time between December 2020 and January 2025. These time points were chosen such that the length of time gave reasonable ability to capture periodic elements but that the size of the data did not become too large. Additional to reduce the size of the data we resample the model simulations to a smaller spatial grid. Figure~\ref{fig:cesm_grid} shows the resampled spatial observation grid over the globe that we use.
 
 \begin{figure}[htbp!] 
 	\centering    
 	\includegraphics[width=1.0\textwidth]{Example_image}
 	\caption[CESM-LE resampled spatial grid]{The resampled spatial grid of observation measurements across the globe.}
 	\label{fig:cesm_grid}
 \end{figure}
 
 
 
  In the following work we  focus on four atmospheric model variables from the  CESM-LE simulations. These are; pressure (see Section~\ref{ssec:pressure}), Temperature (see Section~\ref{ssec:temp}), Precipitation (see Section~\ref{ssec:precip}), and Wind (see Section~\ref{ssec:wind}). We describe each component in detail in their respective section and throughout this work we consider each as a separate EO data set. 

\subsection{Precipitation \label{ssec:precip}}
The total (vertically integrated) precipitable water component abbreviated as TMQ in the model descriptions is an atmospheric component output of the CESM-LE. The component is given units of $\si{\kilogram\per\metre\squared} $ and is available monthly on the full spatial grid with monthly precipitation being average over time from the model 6 hourly output. 

We can see clearly the spatial variability of the precipitation over the globe by considering the heat map of June 2021 monthly precipitation for a single simulation which is shown in Figure~\ref{fig:precip_june}. As one would expect there is clear spatial correlation as for example the tropics observe large amounts of precipitation whereas desert regions observe little. 
 

\begin{figure}[htbp!] 
	\centering    
	\includegraphics[width=1.0\textwidth]{Example_image}
	\caption[Monthly precipitation for June 2021]{The simulated average monthly precipitation across the globe for June 2021 from ensemble member 1 of the CESM-LE data. }
	\label{fig:precip_june}
\end{figure}

We can similarly observe clear temporal correlations in the precipitation variable of the CESM model. In particular Figure~\ref{fig:precip_temp} shows the time series of two locations on the globe. Each exhibit clear periodic signals as wet seasons and dry seasons repeat each year. Additional we see clearly that there is in sense a smooth transition from one month to the next in terms of precipitation value which suggest a FDA approach as discussed in Chapter~\ref{cha:Into} may be relevant. 

\begin{figure}[htbp!] 
	\centering    
	\includegraphics[width=1.0\textwidth]{Example_image}
	\caption[Monthly precipitation between December 2020 and January 2025]{The simulated average monthly precipitation at two points on the globe from ensemble member 1 of the CESM-LE data over five years between December 2020 and January 2025.  }
	\label{fig:precip_temp}
\end{figure}


\subsection{Pressure \label{ssec:pressure}}
The surface pressure component abbreviated as PS in the model descriptions is an atmospheric component output of the CESM-LE. The component is given in units of $\si{\pascal}$ and is available monthly on the full spatial grid with monthly pressure being averaged over time from the model 6 hourly outputs.
\subsection{Temperature \label{ssec:temp}}
The temperature model component abbreviated to TREFHT in the model description is an atmospheric component output of the CESM-LE. The component refers to the average temperature in $\si{\kelvin}$ at the model reference height which is $w\si{\meter}$ above sea level. The average is available monthly with the average being that of the model 6 hourly output for the month. Such a response variable is again available on the full spatial grid of the model. 

\subsection{Wind \label{ssec:wind}}
The wind model component abbreviated to U10 in the model description is an atmospheric component output of the CESM-LE. The component refers to the average wind speed in $\si{\meter\per\second}$ at a height of $10\si{\meter}$. Again the component is available on the full spatial grid and is available as a monthly average over time.
 
 